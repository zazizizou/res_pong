\section{Implementation}
The hardware description of the implemented game pong is divided into four modules. Firstly the video controller, which implements the VGA respectively HDMI interface, provides the functionality to display the video output on a monitor. Additionally the image generator creates all required objects like the ball and paddles as well as their movement. Furthermore the match controller manages the interactions between the objects and the state of the match itself. Lastly the audio output is created by the sound generator module.
	\subsection{Video Controller}
    The video controller provides an interface for the image generator so objects can easily be displayed on a monitor. Thereby it makes the coordinates of the current pixel available and hides all additional requirements of the protocol like timing. The point of origin of the pixel matrix is located at the top left corner of the monitor. Because our group has access to the Atlys Spartan-6 board with HDMI connector as well as the Nexys 4 board with VGA connector, we decided to implement a video controller for both protocols, so the remaining modules could be implemented using both boards. The main focus was laid on the above mentioned uniform interface for the image generator so the remaining development is independent of the used video controller.
        \subsubsection{VGA Controller}
        \subsubsection{HDMI Controller}
    \subsection{Image Generator}
    \subsection{Match Controller}
    \subsection{Sound Generator}
